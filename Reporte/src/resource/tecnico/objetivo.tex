\subsection{El objetivo del programa}
Es desarrollar un sistema de recuperación de información que pueda realizar consultas booleanas en una colección de documentos de texto. Aquí hay algunos puntos clave sobre el objetivo del programa:
\begin{itemize}
  \item Recuperación de Información: El programa está diseñado para permitir a los usuarios realizar consultas para buscar documentos que contengan ciertas palabras o combinaciones de palabras.
  \item Consultas Booleanas: El sistema admite consultas booleanas, lo que significa que los usuarios pueden combinar términos de búsqueda con operadores lógicos como AND, OR y NOT para refinar sus consultas y obtener resultados más relevantes.
  \item Procesamiento de Texto: Antes de ejecutar las consultas, el programa procesa el texto de los documentos y las consultas de varias maneras, como eliminación de caracteres especiales, conversión a minúsculas, eliminación de stopwords, stemming, etc.
  \item Interfaz Gráfica: Proporciona una interfaz gráfica de usuario (GUI) para que los usuarios interactúen con el sistema de manera intuitiva.
  \item Generación de Resultados: Después de ejecutar una consulta, el sistema devuelve una lista de documentos que coinciden con los criterios de búsqueda especificados por el usuario.
  \item Almacenamiento y Manejo de Datos: El sistema maneja la estructura de datos necesaria para almacenar la información sobre los documentos y las consultas, como matrices binarias, tablas hash, etc.
\end{itemize}