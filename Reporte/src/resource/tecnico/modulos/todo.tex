\subsubsection{crear\_tabla}
crear tabla(self, array, nombre). Este metodo crea una tabla a partir de un array y la almacena en un archivo de texto. Es utilizado para almacenar tablas de palabras, frecuencias, stopwords y stemming.
\subsubsection{crear\_matriz\_binaria}
crear matriz binaria(self, palabras, numero documentos). Este metodo crea una matriz binaria donde cada fila representa una palabra y cada columna representa un documento. El valor en cada celda de la matriz indica si la palabra aparece en el documento correspondiente.
\subsubsection{computar\_hash}
computar hash(self, word). Calcula el valor hash MD5 de una palabra y lo devuelve como una cadena hexadecimal. crear hash table(self, words, binary dict). Crea una tabla hash que asocia palabras a sus valores binarios en la matriz binaria.
\subsubsection{token\_snow}
token stopw(self, string). Realiza la tokenizacion, elimina las stopwords y aplica stemming a una cadena de texto.
\subsubsection{procesar bool exp}
procesar bool exp(self, query str). Toma una cadena de consulta booleana y la convierte en una
lista de elementos que representan la expresion booleana.
\subsubsection{procesar postfijo}
procesar postfijo(self, bool exp list). Toma una lista de elementos que representan la expresion
booleana en notacion infija y la convierte en notacion postfija.
\subsubsection{operador or}
or operator(self). Implementa la operacion logica OR en listas binarias y devuelve el resultado.
\subsubsection{operador and}
and operator(self). Implementa la operacion logica AND en listas binarias y devuelve el resultado.
\subsubsection{operador not}
not operator(self). Implementa la operacion logica NOT en una lista binaria y devuelve el resultado.
\subsubsection{esperar consulta}
esperar consulta(self). Bloquea hasta que se recibe una consulta desde la interfaz grafica.
\subsubsection{recibir consulta}
recibir consulta(self, consulta). Recibe una consulta desde la interfaz grafica y la almacena para
su posterior procesamiento.
\subsubsection{procesar query}
process query(self, operator). Ejecuta una operacion booleana (AND, OR o NOT) en los arrays
binarios y devuelve el resultado.
\subsubsection{hash query}
hash query(self, hash). Busca en la tabla hash un valor binario correspondiente a un hash de
palabra.
\subsubsection{ejecutar query}
ejecutar query(self, postfijo). Ejecuta una consulta booleana en notacion postfija utilizando los
arrays binarios y devuelve el resultado.
\subsubsection{obtener docs}
obtener docs(self, array docs). Imprime los documentos que satisfacen una consulta booleana en
funcion del resultado de la evaluacion.
\subsubsection{clear}
clear(self). Limpia las variables y reinicia el sistema para una nueva consulta.
\subsubsection{enviar consulta}
enviar consulta(self). Este metodo se llama cuando se hace clic en el boton ”Enviar”o se presiona
la tecla Enter despues de ingresar una consulta en el cuadro de texto. Toma la consulta ingresada
por el usuario y la emite como una se˜nal consulta signal para que el sistema de recuperacion de
informacion la procese.
\subsubsection{update output}
update output(self, output). Este metodo agrega resultados a la lista de resultados en la interfaz
grafica.
\subsubsection{init interfaz}
init (self, sistema ri). El constructor de la clase InterfazRI inicializa la interfaz grafica de usuario
(GUI) y recibe una instancia de SistemaRI como argumento para establecer una comunicacion entre
la interfaz y el sistema de recuperacion de informacion.
\subsubsection{initUI}
initUI(self). Este metodo configura la interfaz de usuario utilizando la biblioteca PyQt5. Define la
estructura de la ventana principal, que incluye un menu, un area para ingresar consultas, un boton
para enviar consultas, una lista para mostrar resultados y un area para mostrar el contenido del
documento seleccionado.
\subsubsection{openDocument}
openDocument(self). Abre un cuadro de dialogo para seleccionar y abrir un documento de texto.
Luego, muestra el contenido del documento en el area de visualizacion.
\subsubsection{update document displayed}
update document displayed(self, item). Este metodo se llama cuando se hace clic en un elemento de la lista de resultados. Si el elemento hace referencia a un documento, muestra el contenido
del documento seleccionado en el area de visualizacion.
\subsubsection{mostrar ayuda}
mostrar ayuda(self). Muestra un cuadro de dialogo modal (QDialog) con informacion de ayuda
sobre como usar la aplicacion.
\subsubsection{closeEvent}
closeEvent(self, event). Este metodo maneja el evento de cierre de la ventana principal y asegura
que la aplicacion se cierre correctamente cuando se cierra la ventana.
