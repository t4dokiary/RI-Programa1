\subsection{Instalacion}
\begin{itemize}
  \item os: Permite realizar operaciones relacionadas con archivos y directorios.
  \item re: Se utiliza para realizar operaciones con expresiones regulares. unicodedata.normalize: Ayuda a normalizar los caracteres Unicode en texto.
  \item nltk: Es la biblioteca de procesamiento de lenguaje natural. Se utiliza para tokenizar, eliminar palabras vacias y realizar stemming.
  \item numpy: Proporciona funcionalidades para operaciones con matrices y arreglos multidimensionales.
  \item pandas: Se utiliza para el manejo, analisis y procesamiento de datos en forma tabular.
  \item SnowballStemmer: Se utiliza para configurar stemmers para espanol e ingles utilizando Snowball Stemmer. Estos stemmers se utilizan para reducir las palabras a su forma raiz.
  \item hashlib: Permite calcular hash de cadenas de texto.
  \item pyparsing: Usada para definir y analizar gramaticas, especialmente en la definicion de consultas booleanas.
  \item PyQt5: Es una biblioteca de enlace de Python para la biblioteca de interfaz grafica multiplataforma Qt. Se utiliza para crear la interfaz grafica de usuario.
  \item sys: Proporciona funciones y variables que se utilizan para manipular diferentes partes del entorno de ejecucion de Python.
\end{itemize}
Teniendo esta libreras instaladas y posteriormente tener Python y Visual Code, teniendo los plugis de depuracion, se podra realizar la revision del programa.