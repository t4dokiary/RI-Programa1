El objetivo central de esta práctica se orienta hacia el desarrollo de un Sistema de Recuperación de Información (RI) basado en un modelo booleano, el cual se divide en dos partes fundamentales que trabajan en conjunto para lograr su funcionalidad integral.

En la primera fase, se enfoca en la preparación de datos, un proceso esencial donde se realizan diversas operaciones sobre los documentos de texto para extraer información relevante. Esto implica la identificación y almacenamiento de palabras clave significativas, así como la creación de una matriz binaria que representa la presencia o ausencia de cada término en los documentos. Esta etapa establece los cimientos del sistema, proporcionando la estructura necesaria para la búsqueda y recuperación eficiente de información.

La segunda fase se concentra en la creación de una interfaz gráfica de usuario (GUI) que simplifica la interacción de los usuarios con el sistema de RI. A través de esta interfaz intuitiva, los usuarios tienen la capacidad de ingresar consultas booleanas de manera amigable, explorar los resultados de búsqueda de forma visualmente accesible y examinar los documentos recuperados de manera intuitiva. Esta GUI no solo mejora la experiencia del usuario al hacer que el sistema sea más accesible y fácil de usar, sino que también agrega una capa de interactividad que facilita la exploración y comprensión de los resultados.

En conjunto, estas dos partes del código conforman un sistema completo de RI que fusiona el procesamiento de datos subyacente con una interfaz de usuario intuitiva y amigable. Esta integración proporciona a los usuarios una herramienta poderosa y eficiente para buscar y recuperar información en documentos de texto, mejorando así la productividad y la experiencia general del usuario. La combinación de un sólido procesamiento de datos con una interfaz fácil de usar representa un avance significativo en la eficacia y accesibilidad de los sistemas de RI.